\documentclass[11pt]{article}

\usepackage{fullpage}
\usepackage{amsmath, amssymb, bm, cite, epsfig, psfrag}
\usepackage{graphicx}
\usepackage{float}
\usepackage{amsthm}
\usepackage{amsfonts}
\usepackage{listings}
\usepackage{cite}
\usepackage{hyperref}
\usepackage{tikz}
\usepackage{enumerate}
\usepackage{listings}
\usepackage{mathtools}
\lstloadlanguages{Python}
\usetikzlibrary{shapes,arrows}
%\usetikzlibrary{dsp,chains}

\DeclareFixedFont{\ttb}{T1}{txtt}{bx}{n}{9} % for bold
\DeclareFixedFont{\ttm}{T1}{txtt}{m}{n}{9}  % for normal
% Defining colors
\usepackage{color}
\definecolor{deepblue}{rgb}{0,0,0.5}
\definecolor{deepred}{rgb}{0.6,0,0}
\definecolor{deepgreen}{rgb}{0,0.5,0}
\definecolor{backcolour}{rgb}{0.95,0.95,0.92}

%\restylefloat{figure}
%\theoremstyle{plain}      \newtheorem{theorem}{Theorem}
%\theoremstyle{definition} \newtheorem{definition}{Definition}

\def\del{\partial}
\def\ds{\displaystyle}
\def\ts{\textstyle}
\def\beq{\begin{equation}}
\def\eeq{\end{equation}}
\def\beqa{\begin{eqnarray}}
\def\eeqa{\end{eqnarray}}
\def\beqan{\begin{eqnarray*}}
\def\eeqan{\end{eqnarray*}}
\def\nn{\nonumber}
\def\binomial{\mathop{\mathrm{binomial}}}
\def\half{{\ts\frac{1}{2}}}
\def\Half{{\frac{1}{2}}}
\def\N{{\mathbb{N}}}
\def\Z{{\mathbb{Z}}}
\def\Q{{\mathbb{Q}}}
\def\R{{\mathbb{R}}}
\def\C{{\mathbb{C}}}
\def\argmin{\mathop{\mathrm{arg\,min}}}
\def\argmax{\mathop{\mathrm{arg\,max}}}
%\def\span{\mathop{\mathrm{span}}}
\def\diag{\mathop{\mathrm{diag}}}
\def\x{\times}
\def\limn{\lim_{n \rightarrow \infty}}
\def\liminfn{\liminf_{n \rightarrow \infty}}
\def\limsupn{\limsup_{n \rightarrow \infty}}
\def\GV{Guo and Verd{\'u}}
\def\MID{\,|\,}
\def\MIDD{\,;\,}

\newtheorem{proposition}{Proposition}
\newtheorem{definition}{Definition}
\newtheorem{theorem}{Theorem}
\newtheorem{lemma}{Lemma}
\newtheorem{corollary}{Corollary}
\newtheorem{assumption}{Assumption}
\newtheorem{claim}{Claim}
\def\qed{\mbox{} \hfill $\Box$}
\setlength{\unitlength}{1mm}

\def\bhat{\widehat{b}}
\def\ehat{\widehat{e}}
\def\phat{\widehat{p}}
\def\qhat{\widehat{q}}
\def\rhat{\widehat{r}}
\def\shat{\widehat{s}}
\def\uhat{\widehat{u}}
\def\ubar{\overline{u}}
\def\vhat{\widehat{v}}
\def\xhat{\widehat{x}}
\def\xbar{\overline{x}}
\def\zhat{\widehat{z}}
\def\zbar{\overline{z}}
\def\la{\leftarrow}
\def\ra{\rightarrow}
\def\MSE{\mbox{\small \sffamily MSE}}
\def\SNR{\mbox{\small \sffamily SNR}}
\def\SINR{\mbox{\small \sffamily SINR}}
\def\arr{\rightarrow}
\def\Exp{\mathbb{E}}
\def\var{\mbox{var}}
\def\Tr{\mbox{Tr}}
\def\tm1{t\! - \! 1}
\def\tp1{t\! + \! 1}
\def\Tm1{T\! - \! 1}
\def\Tp1{T\! + \! 1}


\def\Xset{{\cal X}}

\newcommand{\one}{\mathbf{1}}
\newcommand{\abf}{\mathbf{a}}
\newcommand{\bbf}{\mathbf{b}}
\newcommand{\dbf}{\mathbf{d}}
\newcommand{\ebf}{\mathbf{e}}
\newcommand{\gbf}{\mathbf{g}}
\newcommand{\hbf}{\mathbf{h}}
\newcommand{\pbf}{\mathbf{p}}
\newcommand{\pbfhat}{\widehat{\mathbf{p}}}
\newcommand{\qbf}{\mathbf{q}}
\newcommand{\qbfhat}{\widehat{\mathbf{q}}}
\newcommand{\rbf}{\mathbf{r}}
\newcommand{\rbfhat}{\widehat{\mathbf{r}}}
\newcommand{\sbf}{\mathbf{s}}
\newcommand{\sbfhat}{\widehat{\mathbf{s}}}
\newcommand{\ubf}{\mathbf{u}}
\newcommand{\ubfhat}{\widehat{\mathbf{u}}}
\newcommand{\utildebf}{\tilde{\mathbf{u}}}
\newcommand{\vbf}{\mathbf{v}}
\newcommand{\vbfhat}{\widehat{\mathbf{v}}}
\newcommand{\wbf}{\mathbf{w}}
\newcommand{\wbfhat}{\widehat{\mathbf{w}}}
\newcommand{\xbf}{\mathbf{x}}
\newcommand{\xbfhat}{\widehat{\mathbf{x}}}
\newcommand{\xbfbar}{\overline{\mathbf{x}}}
\newcommand{\ybf}{\mathbf{y}}
\newcommand{\zbf}{\mathbf{z}}
\newcommand{\zbfbar}{\overline{\mathbf{z}}}
\newcommand{\zbfhat}{\widehat{\mathbf{z}}}
\newcommand{\Ahat}{\widehat{A}}
\newcommand{\Abf}{\mathbf{A}}
\newcommand{\Bbf}{\mathbf{B}}
\newcommand{\Cbf}{\mathbf{C}}
\newcommand{\Bbfhat}{\widehat{\mathbf{B}}}
\newcommand{\Dbf}{\mathbf{D}}
\newcommand{\Gbf}{\mathbf{G}}
\newcommand{\Hbf}{\mathbf{H}}
\newcommand{\Ibf}{\mathbf{I}}
\newcommand{\Kbf}{\mathbf{K}}
\newcommand{\Pbf}{\mathbf{P}}
\newcommand{\Phat}{\widehat{P}}
\newcommand{\Qbf}{\mathbf{Q}}
\newcommand{\Rbf}{\mathbf{R}}
\newcommand{\Rhat}{\widehat{R}}
\newcommand{\Sbf}{\mathbf{S}}
\newcommand{\Ubf}{\mathbf{U}}
\newcommand{\Vbf}{\mathbf{V}}
\newcommand{\Wbf}{\mathbf{W}}
\newcommand{\Xhat}{\widehat{X}}
\newcommand{\Xbf}{\mathbf{X}}
\newcommand{\Ybf}{\mathbf{Y}}
\newcommand{\Zbf}{\mathbf{Z}}
\newcommand{\Zhat}{\widehat{Z}}
\newcommand{\Zbfhat}{\widehat{\mathbf{Z}}}
\def\alphabf{{\boldsymbol \alpha}}
\def\betabf{{\boldsymbol \beta}}
\def\betabfhat{{\widehat{\bm{\beta}}}}
\def\epsilonbf{{\boldsymbol \epsilon}}
\def\mubf{{\boldsymbol \mu}}
\def\lambdabf{{\boldsymbol \lambda}}
\def\etabf{{\boldsymbol \eta}}
\def\xibf{{\boldsymbol \xi}}
\def\taubf{{\boldsymbol \tau}}
\def\sigmahat{{\widehat{\sigma}}}
\def\thetabf{{\bm{\theta}}}
\def\thetabfhat{{\widehat{\bm{\theta}}}}
\def\thetahat{{\widehat{\theta}}}
\def\mubar{\overline{\mu}}
\def\muavg{\mu}
\def\sigbf{\bm{\sigma}}
\def\etal{\emph{et al.}}
\def\Ggothic{\mathfrak{G}}
\def\Pset{{\mathcal P}}
\newcommand{\bigCond}[2]{\bigl({#1} \!\bigm\vert\! {#2} \bigr)}
\newcommand{\BigCond}[2]{\Bigl({#1} \!\Bigm\vert\! {#2} \Bigr)}
\newcommand{\tran}{^{\text{\sf T}}}
\newcommand{\herm}{^{\text{\sf H}}}
\newcommand{\bkt}[1]{{\langle #1 \rangle}}
\def\Norm{{\mathcal N}}
\newcommand{\vmult}{.}
\newcommand{\vdiv}{./}


\def\hid{\textsc{\tiny H}}
\def\out{\textsc{\tiny O}}

% Python style for highlighting
\newcommand\pythonstyle{\lstset{
language=Python,
backgroundcolor=\color{backcolour},
commentstyle=\color{deepgreen},
basicstyle=\ttm,
otherkeywords={self},             % Add keywords here
keywordstyle=\ttb\color{deepblue},
emph={MyClass,__init__},          % Custom highlighting
emphstyle=\ttb\color{deepred},    % Custom highlighting style
stringstyle=\color{deepgreen},
%frame=tb,                         % Any extra options here
showstringspaces=false            %
}}

% Python environment
\lstnewenvironment{python}[1][]
{
\pythonstyle
\lstset{#1}
}
{}

% Python for external files
\newcommand\pythonexternal[2][]{{
\pythonstyle
\lstinputlisting[#1]{#2}}}

% Python for inline
\newcommand\pycode[1]{{\pythonstyle\lstinline!#1!}}

\begin{document}

\title{Introduction to Machine Learning\\
Homework 7:  Neural Networks}
\author{Prof. Sundeep Rangan and Yao Wang}
\date{}

\maketitle

% Submit answers to only problems 1--3.  But, make sure you know how to do all problems.

\begin{enumerate}



\item Consider a neural network
on a 3-dimensional input $\xbf=(x_1,x_2, x_3)$ with weights and biases:
\[
    W^\hid = \left[ \begin{array}{ccc} 1 & 0 & 1  \\ 0 & 1  & 1 \\ 1 & 1 & 0 \\ 1 & 1 & 1\\ \end{array} \right], \quad
    b^\hid = \left[ \begin{array}{c} 0 \\ 0 \\ -1 \\ 1 \end{array} \right] \quad
    W^\out = [1, 1, -1, -1], \quad b^\out = -1.5.
\]
Assume the network uses the
threshold activation function \eqref{eq:gact_ht}
\beq \label{eq:gact_ht}
    g_{\rm act}(z) = \begin{cases}
        1, & \mbox{if } z \geq 0 \\
        0, & \mbox{if } z < 0.
        \end{cases}
\eeq
 and the
threshold output function \eqref{eq:gout_thresh}:
\beq \label{eq:gout_thresh}
    \hat{y} =  \begin{cases}
        1 & z^\out \geq 0 \\
        0 & z^\out < 0.
        \end{cases}
\eeq



\begin{enumerate}[(a)]
\item Write the components of $\zbf^\hid$ and $\ubf^\hid$ as a function
of $(x_1,x_2, x_3)$.  For each component $j$, indicate where in the $(x_1,x_2, x_3)$
hyperplane $u^\hid_j=1$.

\item Write $z^\out$ as a function of $(x_1,x_2, x_3)$.  In what region is
$\hat{y}=1$? (You can describe in mathematical formulae).
\end{enumerate}



\item Consider a neural network used for regression with a scalar input $x$ and
scalar target $y$,
\begin{align*}
    & z_{j}^\hid = W^\hid_{j}x + b^\hid_j, \quad
    u_{j}^\hid = \max\{0, z^\hid_j\},
    \quad j=1,\ldots,N_h \\
    &
    z^\out = \sum_{k=1}^{N_h} W^\out_{k}u^\hid_{k} + b^\out,
    \quad
    \hat{y} = g_{\rm out}(z^\out).
\end{align*}
The hidden weights and biases are:
\[
    \Wbf^\hid = \left[ \begin{array}{c} -1  \\ 1  \\  1 \end{array} \right], \quad
    \bbf^\hid = \left[ \begin{array}{c} -1 \\ 1 \\ -2 \end{array} \right].
\]

\begin{enumerate}[(a)]

\item What is the number $N_h$ of hidden units?  For each $j=1,\ldots,N_h$,
draw $u^\hid_j$ as a function of $x$ over some suitable range of values $x$. 
You may draw them on one plot, or on
multiple plots.

\item Since the network is for regression, you may choose 
the activation function $g_{\rm out}(z^\out)$ to be linear.
Given training data $(x_i,y_i)$, $i=1,\ldots,N$, formally define the loss function you would use
to train the network?

\item Using the output activation and loss function selected in part (b), set up the formulation to determine 
output weights and bias, $\Wbf^\out$, $b^\out$,
for the training data below.  You should be able to find a closed form solution. Write a few line of python code to solve the problem.
\begin{center}
\begin{tabular}{|c|c|c|c|c|c|} \hline
$x_i$ & -2 & -1 & 0 & 3 & 3.5\\ \hline
$y_i$ & 0 & 0   & 1 & 3 & 3 \\ \hline
\end{tabular}
\end{center}

\item
 Based on your solution for the output weights and bias,  
draw $\hat{y}$ vs.\ $x$ over some suitable range of values $x$.  Write a few line of python code to do this.



\item Write a function \pycode{predict} to output $\hat{y}$ given a vector of inputs
\pycode{x}.  Assume \pycode{x} is a vector representing a batch of samples and
\pycode{yhat} is a vector with the corresponding outputs.  Use the activation function
you selected in part (b), but your function should take the weights and biases
for both layers as inputs.  Clearly state any assumptions on the formats for
the weights and biases.  Also, to receive full credit, you must not use any for loops.
\end{enumerate}

\pagebreak
\item \label{prob:cg_exp}
Consider a neural network that takes each input $\xbf$ and produces
a prediction $\hat{y}$ given
\begin{align} \label{eq:nnmax}
\begin{split}
    z_j  &= \sum_{k=1}^{N_i}W_{jk}x_k + b_j, \quad u_j = 1/(1+\exp(-z_j)),
    \quad j = 1,\ldots,M,\\
    \hat{y} &= \frac{ \sum_{j=1}^{M} a_j u_j}{ \sum_{j=1}^M u_j},
\end{split}
\end{align}
where $M$ is the number of hidden units and is fixed (i.e.\ not trainable).
To train the model, we get training data $(\xbf_i,y_i)$, $i=1,\ldots,N$.
\begin{enumerate}[(a)]
\item Rewrite the equations \eqref{eq:nnmax} for the
batch of inputs $\xbf_i$ from the training data.  That is, correctly
add the indexing $i$, to the equations.

\item If we use a loss function,
\[
    L = \sum_{i=1}^N (y_i - \hat{y}_i)^2,
\]
draw the computation graph describing the mapping from $\xbf_i$ and parameters 
to $L$.
Indicate which nodes are trainable parameters.


\item Compute the gradient  $\partial L/\partial \hat{y}_i$
for all $i$.  

\item Suppose that, in backpropagation, we have computed $\partial L/\partial \hat{y}_i$
for all $i$, represented as $\partial L/\partial \hat{\bf y}$. Describe how to compute the components of the gradient
$\partial L/\partial \ubf$.

\item
Suppose that we have computed the gradient $\partial L/\partial \ubf$, describe how would you compute the gradient $\partial L/\partial \zbf$.



\item
Suppose that we have computed the gradient $\partial L/\partial \zbf$, describe how would you compute the gradient
 $\partial L/\partial W_{jk}$ and $\partial L/\partial b_j$.
 
 \item
 Put all above together, describe how would you compute the gradient  $\partial L/\partial W_{jk}$ and $\partial L/\partial b_j$.
 

\item Write a few lines of python code to implement the gradients  $\partial L/\partial \ubf$ (as in part (d)), given the gradient $\partial L/\partial \hat{\bf y}$.
Indicate how you represent the gradients.
Full credit requires that you avoid for-loops.
 
\end{enumerate}

\end{enumerate}
\end{document}

c